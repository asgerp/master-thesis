%!TEX root = report.tex
\chapter{Related Work} % (fold)
\label{cha:related_work}
[Introduction to related work. ]

Research in moving controls onto the physical desktop has mainly focused on four approaches; extending the screen onto the desktop using projection, using a physical surface as an input space, pen tracking on paper and multi-touch desktops.



\subsection{Augmented and interactive desktops} % (fold)
\label{sub:aug}
% more here!!

%Augmented and interactive desktops
Research in the field of projection and interaction on physical surfaces was pioneered by Wellner’s DigitalDesk\cite{wellner1993interacting}. Wellner saw a problem in what he called "The dual desk", that is one physical and one digital desktop, each supporting separate interaction styles and functionalities, and both with different strengths and weaknesses. Physically interacting with digital documents on the digital desktop was limited compared to how we could interact with pencils, paper and other tools in the physical world, and while digital documents had direct access to tools such as spell-checking, they did not have the tangibility and portability of their physical counterparts. These differing interaction styles and functionality sets combined with the lack of integration between the physical and digital desktop motivated the development of a hybrid solution with the best of both called DigitalDesk.\\
The system was able to project a digital interface onto a surface, and would enable users to interact with the system by tracking their fingers and could understand information on physical paper.

Rekimoto and Saitoh\cite{rekimoto1999augmented}, wants to enable users to smoothly interchange information between walls, tables and portable computers. Portable computers are plentiful and used by many as their primary computing device,and large surfaces and multiple displays are essential for collaborative and in some cases individual activities. Rekimoto and Saitoh therefore intend to support interaction between personal and public computing activities. \\
To support this interaction they created a computer augmented environment where users can easily transfer information from their personal portable computers, onto public computerised walls and tables. Users can use displays projected on tables and walls as a spatially continuous extension of their portable computers. They also provide mechanism for attaching digital data to physical objects.


Kane et al. \cite{kane2009bonfire} saw the growing availability of interactive tabletop systems and ubiquity of laptop computers and explored the intersection. Their contribution involved the observation of the adjacent space providing integration between physical and digital objects, extending the laptop display onto an adjacent physical space allowing interaction via pointing and gestures and enlarging the input/output space to enrich existing applications. The system named BonFire therefore combined previously documented features, but packaged them in a new nomadic computer platform.

%BonFire \cite{kane2009bonfire} is a self contained mobile computing system that consists of two laptop-mounted micro-projectors and accompanying cameras. The projectors are used to project an interactive display on the empty space on either side of a laptop, while using the cameras to perform hand gesture tracking, object recognition, and information transfer from the projected space. The system recognises physical objects placed in its line of sight, and enables integration between physical and digital objects.

%Multi-touch displays
\subsection{Multi-touch displays} % (fold)
\label{sub:multi_touch_displays}

Recent years have seen much interest in multi-touch displays, and research has shown many benefits of multi-touch input. Most of the research has been done on stand-alone multi-touch devices, leaving the field of integrating multi-touch with the standard desktop computing experience. Bi et al.\cite{bi2011magic}, foresee that future computing environments include keyboards, mice, and multi-touch surfaces.\\
Magic Desk\cite{bi2011magic} attempts to integrate multi-touch input on a digital multi-touch surface, with a mouse and a keyboard to facilitate desktop work. In particular the authors demonstrate how different regions within the desktop environment can be used for touch. They also shed light on what types of input is best suited for each region. Bi et al. in particular investigate how using the regions around the keyboard compare the standard vertical touch  input interface. The main focus with Magic Desk is on enhancing window management and taskbar functionality and providing a multi-functional touch pad.
%MORE
% subsection multi_touch_displays (end)


% subsection projection_and_multi_touch_input_on_digital_surfaces (end)
\subsection{Using the physical world as input canvas with minimal feedback} % (fold)
\label{sub:use_desktop_as_input_canvas}

%Using the physical world as input canvas with minimal feedback
Limited screen real estate means that some information, such as menus, must be hidden and that information require effort to retrieve. The problem is worsened by smaller screens in laptops and tablet computers. Hauser et al. investigate how we can utilize the physical desktop as an additional input canvas without any augmentation. User can interact with an area of their desk, and the interaction is used as input for a computer. Users can quickly store, organize and select items from this space with optional feedback on the screen. \\
Hauser et al. found that participants freely used discrete areas on the desk and that the participant organised items in grid shapes for easier access. When fewer, and therefore larger, items were located in the area the participants made fewer errors, and they increasingly needed feedback when the number of items grew, and therefore were smaller.


% subsection use_desktop_as_input_canvas (end)


\subsection{Using pen and paper} % (fold)
\label{sub:using_pen_and_paper}
% VoodooSketch
%Using pen and paper
The VoodooSketch system sets out to separate graphical user interface elements from the  interactive work surface and provides tools, in the shape of a physical interface palette, that can be custom-assembled depending on the current set of tasks. Application functionality can be bound to tangible controls or sketches on the physical palette and controlled without needing to interact with the underlying graphical interface of the application. \\
Block et al. identified four challenges to existing large interactive surfaces as motivation; (1) Interface localization -- where moving between activities and graphical elements is time-consuming, (2) Collaboration -- where the interface needs to support multiple simultaneous users that might share a set of interface components, (3) Different perspectives -- where different perspectives mean interface elements must adapt to different views, (4) Adaptability -- where different users may have different requirements, leading to a need for customizable controls depending on the user and task. \\
The VoodooSketch system\cite{Block:2008:VEI:1347390.1347404} is designed to extend large interactive surfaces with physical interface palettes which consist of sketched controls on paper with tangible controls. The physical palette itself leverages the Anoto\cite{anoto2011anoto} system for its sketched controls where handwriting recognition is used to link sketched controls with application functionality. Additionally, the physical palette utilises the VoodooIO\cite{villar2007voodooio} system for its tangible controls, such as joysticks, sliders and buttons.


% PLink
Steimle et al. concluded that although an increasing part of our activities involve digital media, the pen and paper is still relevant. They saw an opportunity existed in utilising the paper already surrounding the computer in the shape of post-it notes, notes and sketches and wanted to combine these with digital media. The application they developed called PLink \cite{steimle2011plink} utilised an Anoto pen to give users the ability to sketch shortcuts, that could later be interacted with,  on a paper surface. PLink’s shortcuts were limited to references into digital documents, starting applications, and opening files. \\
An initial study revealed that PLink's paper links are effective in supporting users organising digital resources during web searches.


% integrations between phys and dig desktop
A longer study, 4 weeks, showed that user made paper links for directly and rapidly accessing resources, and personal "portals" that linked to files, folders and programs. Portals were created on the users dominant hand side and normally not occluded. \\
%spacial
The users ordered the link topically instead of in linear sequences. And when the users were performing information gathering tasks, they externalized less information when they used PLink as the content of the resource could be retrieved quickly. \\
%temporal
Generally the usage of the links decrease temporally, usually within hours, yet Steimle et al. saw some user show the opposite trend, where their usage rose even after 7 days. \par
% Study showed that
% PaperPoint

PaperPoint \cite{Signer:2007:PPP:1226969.1226981} utilises an Anoto\cite{anoto2011anoto} pen and interactive paper to control and annotate PowerPoint  presentations. The slides are printed on interactive paper and the presenter can then chose which slides are shown. The presenter can also annotate the slides during the presentation, by using the Anoto pen on the printed slides, making the annotations appear digitally. \\

Users reported that they liked the flexibility provided by PaperPoint and that using the system has changed the way they prepared presentations; they started to include slides in their presentations that they were not completely sure would be used. This only seemed to be an advantage when the presenter were familiar with the slides or the presenter was experienced, for others the usage of PaperPoint resulted in presenters eing under-prepared in structure and content of the presentation. The PaperPoint system were also reported to work well when more than one presenter used the system.\\

\par



% subsection using_pen_and_paper (end)



[Sum up was the related work is and relate and contrast to what we are going to do]


%Overall it is clear that many attempts have been made to move functionality away from the main screen and onto an adjacent medium. We know we can utilise the empty space on the desktop from the work done on BonFire \cite{kane2009bonfire}, MagicDesk \cite{bi2011magic} and the Unadorned Desk \cite{hausen2012unadorned} especially. PLink \cite{steimle2011plink} allows the user to have more than one piece of digital paper, even special dotted post-its. We will also use more than one piece of paper to give users the possibility of putting together several papers at will to create personalised controls. The focus of VoodooSketch \cite{Block:2008:VEI:1347390.1347404} is on enabling round-the-table collaborative adaptable controls, but for a single application. Like VoodooSketch , we will also focus on controller adaptability and creation, however we will focus on a single user, in a computer-desktop configuration. We will focus on creating and using controls for multiple applications when they are needed, whereas VoodooSketch and PaperPoint \cite{Signer:2007:PPP:1226969.1226981} focuses on creating or using controls configured for one application. In short we will combine utilisation of empty space around the desktop investigated by MagicDesk, BonFire, Unadorned Desk, and using multiple papers for controls as investigated in PLink. Furthermore, we will provide a better understanding of ad hoc creation of controls on paper for single users.
% chapter related_work (end)


